\addcontentsline{toc}{chapter}{Introduzione}
\markboth{Introduzione}{Introduzione} 
\chapter*{Introduzione} 
L'utilizzo dell'Intelligenza Artificiale in ambito industriale sta prendendo piede negli ultimi anni e il caso studiato in questa tesi ne è la prova. Lo sviluppo della tecnologia ha reso disponibile sempre più potenza computazionale a minor prezzo, rendendo possibile l'utilizzo delle Reti Neurali Profonde, studiate fin dagli anni ottanta, in un modo che fino a non molti anni fa era economicamente insostenibile. Si andrà a vedere il caso concreto della realizzazione di un sistema che esegue previsioni in tempo reale su telemetrie di un impianto per la gestione delle acque, con lo scopo di assistere gli operatori nelle decisioni critiche da prendere in situazioni che potrebbero portare a un'emergenza. Sono state utilizzate tecniche allo stato dell'arte del Deep Learning per la realizzazione della rete previsionale, soluzioni di Big Data e Cloud Computing per la raffinazione dei dati grezzi e rendere possibile il training della rete neurale. Sono state studiate le basi teoriche richieste per realizzare un sistema in streaming, è stata poi progettata e realizzata una architettura apposita dedicata alla trasformazione in tempo reale dei dati per poter realizzare previsioni aggiornate.  \newline Questo documento introdurrà il problema generale nel primo capitolo, nel secondo verranno introdotti più nel dettaglio i dati a disposizione e l'architettura previsionale. Sul terzo capitolo viene illustrato il metodo di trasformazione dei dati per renderli adatti al training della rete previsionale, per poi vedere i risultati dopo il suo addestramento. Nel quarto capitolo verranno introdotti i principi teorici essenziali richiesti per la trasformazione dei dati in tempo reale, con un esempio di riferimento applicato alle diverse tecniche disponibili. Nell'ultimo capitolo viene spiegato il caso d'uso rilevante affrontato dal candidato, andando nel dettaglio della progettazione e implementazione, i risultati ottenuti e i possibili cambiamenti futuri.