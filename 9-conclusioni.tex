\chapter*{Conclusioni}\label{conclusioni}
Nella realizzazione di un sistema che utilizza tecniche di Machine Learning e, in particolare, Deep Learning, un compito essenziale che non va trascurato è quello della trasformazione dei dati. Infatti, buona parte delle informazioni non possono essere utilizzate nella forma originale in cui vengono prodotte. Si è visto il caso della realizzazione di un sistema che preveda l'andamento futuro su telemetrie di un impianto per la gestione delle acque, sfruttando sia i dati storici sia il meteo. Sono stati sfruttati i dati relativi alle stagioni passate per il training della rete, studiata appositamente per prevedere l'evoluzione futura di queste serie temporali. Prima di ottenere il dataset, è stato necessario eseguire diverse trasformazioni, sfruttando tecnologie allo stato dell'arte di Big Data e Cloud Computing. Questo ha permesso di accorciare drasticamente il tempo di trasformazione e training, rendendo molto più rapida la valutazione della bontà del lavoro svolto per prendere eventuali azioni di miglioramento. Le tecnologie utilizzate si sono rivelate indispensabili anche per rendere disponibili i risultati ottenibili tramite il processo di previsione in tempo utile per l'utilizzatore, che adopererà tali informazioni per intraprendere decisioni critiche. La progettazione e realizzazione di questa architettura in tempi brevi richiede una separazione fra la logica di trasformazione e il processo applicato, distinguendo nettamente fra divisione temporale e semantica dei dati. Tale separazione si è resa possibile utilizzando un sistema realizzato specificatamente per lo scopo, dove le tecniche di gestione temporale delle informazioni sono distinte dalle trasformazioni applicate su di esse. Tutti i metodi e le tecnologie utilizzate, da Tensorflow per implementare la rete neurale a Beam per le trasformazioni sui dati, nonchè le tecnologie di Cloud computing di Google Cloud Platform, sono in rapida evoluzione, insieme alla nascita di nuove applicazioni industriali che richiedono di farne uso. % Molti sistemi già esistenti possono essere trasferiti e potenziati su piattaforme di questo tipo che sfruttano il Cloud Computing, purchè sia stata fatta fin da subito separazione fra la business logic e la tecnologia stessa. Con i sistemi visti in questo documento la separazione fra la responsabilità di distribuzione del carico e logica di trasformazione è quasi intrinseca, su sistemi distribuiti dove vi è maggior controllo sul sistema tale divisione non è scontata.